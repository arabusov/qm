\documentclass[t,aspectratio=169]{beamer}

\usetheme{Goettingen}
\setbeamertemplate{itemize items}{ }
\usepackage[T2A]{fontenc}
\usepackage[utf8]{inputenc}
\usepackage[russian]{babel}

\usepackage{amsmath}
\usepackage{bbm}
\usepackage{verbatim}
\usepackage{multimedia}
\usepackage[absolute,overlay]{textpos}

\usepackage{braket}
\usepackage{amssymb}
\usepackage{bbold}

\usepackage{pgfplots}
\usetikzlibrary{3d}
\usepackage{tikz}
\usepackage{tikz-feynman}
\usepackage{tikz-3dplot}
\usetikzlibrary{arrows,automata,backgrounds,calendar,chains,matrix,mindmap,
    patterns,petri,shadows,shapes.geometric,shapes.misc,spy,trees,calc}

\newcommand{\Sp}[1]{\mathbbm {Sp} \left\{ {#1} \right\} }
\usepackage{commath}
%\newcommand\norm[1]{\left\lVert#1\right\rVert}
\setbeamertemplate{navigation symbols}{}
\setbeamertemplate{caption}{\raggedright\insertcaption\par}
\usepackage{url}
\usefonttheme[onlymath]{serif}
\graphicspath{{./images/}}

\author[А. Рабусов]{Андрей Рабусов}
\title[Квантовая физика]{Квантовая физика}
\subtitle{Введение}
\date{27 декабря 2020}
\begin{document}
\begin{frame}
    \maketitle
\end{frame}
%\section{Квантовая механика}
%\subsection{Введение}
\begin{frame}
    \frametitle{Математическая база}
    \begin{itemize}[<+->]
    \item
        Поле комплексных чисел $\mathbb{C}$
        \begin{itemize}[<+->]
        \item $i^2 = -1$
        \item $z = x + i\,y$, $z^\ast = x - i\, y$
        \item $e^{i \varphi} = \cos \varphi + i \sin \varphi$
        \end{itemize}
    \item
        Гильбертово пространство $\mathcal{H}$
        над полем комплексных чисел $\mathbb{C}$
        \begin{itemize}[<+->]
        \item Линейное векторное пространство
        \item
            Определено скалярное произведение для любой пары
            векторов $\bf{u}$ и $\bf{v}$:
            $$(\bf{u}^\dagger, \bf{v}) \in \mathbb{C}$$
            $\bf{u}^\dagger = \left(\bf{u}^\intercal \right)^\ast$
        \item
            Скалярное произведение порождает норму, например
            $$\norm {\bf{u}}^2 = (\bf {u}^\dagger, \bf{u})$$
        \item Обозначения Дирака:
        $\bra{u} = \bf{u}^\dagger$, $\ket{u} = \bf{u}$
        $$\braket{u|v} = (\bf{u}^\dagger, \bf{v})$$
        \end{itemize}
    \end{itemize}
\end{frame}

\subsection{Статические постулаты}
\begin{frame}
    \frametitle{Постулаты $1$--$2$}
    \begin{itemize}[<+->]
    \item[П1] Чистое состояние системы $\ket{\psi}$ описывается
    вектором гильбертова пространства $\mathcal{H}$ над $\mathbb{C}$
    \item[П2] Чистое состояние $\ket{\psi}$ задаётся
    только направлением вектора из $\mathcal{H}$
        \begin{itemize}[<+->]
        \item Можно выбрать любую ненулевую нормировку, например
        $$ \braket{\psi|\psi} = 1 $$
        \item Эволюция замкнутой системы унитарна
        $$ \ket{\psi (t)} = U (t, t_0) \ket{\psi (t_0)}$$
        $$ U^\dagger (t, t_0) \; U (t, t_0) = \hat{\mathbb{1}} $$
        \end{itemize}
    \end{itemize}
\end{frame}
\begin{frame}
    \frametitle{Принцип суперпозиции (постулат 3)}
    \begin{itemize}[<+->]
    \item Положим, что квантовая система до измерения в состоянии
    $\ket{\psi}$
    \item После многократных измерений при помощи макроприбора
    $\ket{\psi} \to \ket{\varphi_i}$
    \item $\ket{\varphi_i}$ различимы при помощи макроприбора, т.~е.
    $$ \braket {\varphi_i| \varphi_j} = \delta_{ij} = 
        \begin{cases}
        0 & \text {if $i \ne j $} \\
        1 & \text {if $i = j $}
        \end{cases}
    $$
    \item
    $ \ket {\psi} = \sum\limits_i c_i \ket {\varphi_i},\;c_i \in \mathbb{C} $
    \item[$\hat{\mathbb{P}}_{\varphi_i}$] Проекционный оператор
    $\hat{\mathbb{P}}_{\varphi_i} = \ket{\phi_i}\bra{\phi_i}$
    $$\hat{\mathbb{P}}_{\varphi_i}\ket{\psi} = \ket{\phi_i}$$
    \item $\sum\limits_i \hat{\mathbb{P}}_{\varphi_i} = \hat{\mathbb{1}}$
    \end{itemize}
\end{frame}

%\subsection{Двухщелевой опыт}
\begin{frame}
    \frametitle{Интерференция}
    \begin{figure}
    % Taken from here:
% https://tex.stackexchange.com/questions/469109/\
% how-to-change-arrowheads-to-lie-on-a-plane
\begin{tikzpicture}[scale=1.25,every node/.append style={transform shape}]
\foreach \x in {-1,-0.75,...,0} {
	\draw (\x,-1) -- (\x,1);
}
\draw[fill=black!10] (0.5,-2,-1) -- (0.5,-2,1) -- (0.5,2,1) -- (0.5,2,-1) -- (0.5,-2,-1);
\fill (0.5,0,0) circle (0.05);
\foreach \r in {0.25,0.5,...,1.75} {
	\draw (0.5,0) ++(-60:\r) arc (-60:60:\r);
}
\draw[fill=black!10] (2,-2,-1) -- (2,-2,1) -- (2,2,1) -- (2,2,-1) -- (2,-2,-1);
%\fill (2,0.5) circle (0.05) (2,-0.5) circle (0.05);
\foreach \r in {0.25,0.5,...,2} {
	\draw (2,0.5) ++(-60:\r) arc (-60:60:\r);
	\draw (2,-0.5) ++(-60:\r) arc (-60:60:\r);
}
\draw[fill=black!10] (4,-2,-1) -- (4,-2,1) -- (4,2,1) -- (4,2,-1) -- (4,-2,-1);
%       LABELLING
\begin{scope}[canvas is yz plane at x=2,rotate=-90]
\node[circle,inner sep=0.5mm,fill,label=above:{S${}_1$}] at (0,0.5){};
\node[circle,inner sep=0.5mm,fill,label=below:{S${}_2$}] at (0,-0.5) {};
\draw[dash pattern=on 1.5pt off 1pt,thin] (-0.5,0.5) -- (0,0.5)
	(-0.5,-0.5) -- (0,-0.5);
	\pgflowlevelsynccm          
	\draw[|<->|] (-0.5,0.5) -- (-0.5,-0.5) node[midway,right=-0.1cm] {d};
\end{scope}
\begin{scope}[canvas is yz plane at x=0.5,rotate=-90]
\node[below left=-0.1cm] at (0,0) {S${}_0$};
\end{scope}
\begin{scope}[xshift=4cm,yshift=2cm,rotate=-90,canvas is xy plane at z=0]
\fill[white] (0,0) rectangle (4,4);
\begin{axis}[
	width=5.575cm,
	xmin=-0.5,
	xmax=0.5,
	ticks=none
]
\addplot [samples=1000,blue
]
{(cos(deg(5*pi*sin(deg(x)))))^(2)*((sin(deg(4*pi*sin(deg(x)))))/(4*pi*sin(deg(x))))^(2)};
\end{axis}
\end{scope}
\draw[thin,densely dashed,blue] (2,0) -- (6.9,0);
\draw[thin,densely dashed,blue] (2,0) -- +(15:2.5);
\draw[thin,densely dashed,blue] (2,0) -- +(-15:2.5);
\draw[thin,densely dashed,blue] (2,0) -- +(32:2.5);
\draw[thin,densely dashed,blue] (2,0) -- +(-32:2.5);
\end{tikzpicture}

    \end{figure}
\end{frame}
\begin{frame}
    \frametitle{Дополнительный детектор у щели $A$ --- дифракция}
    \begin{figure}
    % Taken from here:
% https://tex.stackexchange.com/questions/469109/\
% how-to-change-arrowheads-to-lie-on-a-plane
\begin{tikzpicture}[scale=1.25]%,every node/.append style={transform shape}]
\draw[color=white] (-2,-2) -- (8, 3);
\only<1->{
\fill (0.5,0,0) circle (0.05);
\foreach \r in {0.25,0.5,...,1.75} {
	\draw (0.5,0) ++(-60:\r) arc (-60:60:\r);
\node[below left=-0.1cm] at (0,0) {$S$};
}
}
\only<2->{
\draw[fill=black!10] (2,-2,-1) -- (2,-2,1) -- (2,2,1) -- (2,2,-1) -- (2,-2,-1);
\begin{scope}[canvas is yz plane at x=2,rotate=-90]
\node[circle,inner sep=0.5mm,fill,label=above:{$A\;D$}] at (0,0.5){};
\node[circle,inner sep=0.5mm,fill,label=below:{$B$}] at (0,-0.5) {};
\end{scope}
}
\only<3->{
\foreach \r in {0.25,0.5,...,2} {
	\draw (2,0.5) ++(-60:\r) arc (-60:60:\r);
	\draw (2,-0.5) ++(-60:\r) arc (-60:60:\r);
}
}
\only<4->{
\draw[fill=black!10] (4,-2,-1) -- (4,-2,1) -- (4,2,1) -- (4,2,-1) -- (4,-2,-1);
}
\only<5->{
\begin{scope}[xshift=4cm,yshift=2cm,rotate=-90,canvas is xy plane at z=0]
\fill[white] (0,0) rectangle (4,4);
\begin{axis}[
	width=5.575cm,
	xmin=-0.5,
	xmax=0.5,
	ticks=none
]
\addplot [samples=1000,blue
]
%{(cos(deg(5*pi*sin(deg(x)))))^(2)*((sin(deg(4*pi*sin(deg(x)))))/(4*pi*sin(deg(x))))^(2)};
{((1/((x-0.2)^(2)+0.001))+(1/((x+0.2)^(2)+0.001)))};
\end{axis}
\end{scope}
\draw[thin,densely dashed,blue] (2,0) -- (6.9,0);
}
\end{tikzpicture}

    \end{figure}
\end{frame}
\begin{frame}
    \frametitle{Волновое поведение}
    \begin{figure}
    % Taken from here:
% https://tex.stackexchange.com/questions/469109/\
% how-to-change-arrowheads-to-lie-on-a-plane
\begin{tikzpicture}[scale=1.25]%,every node/.append style={transform shape}]
\draw[color=white] (-2,-2) -- (8, 3);
\fill (0.5,0,0) circle (0.05);
\foreach \r in {0.25,0.5,...,1.75} {
	\draw (0.5,0) ++(-60:\r) arc (-60:60:\r);
\node[below left=-0.1cm] at (0,0) {$S$};
}
\draw[fill=black!10] (2,-2,-1) -- (2,-2,1) -- (2,2,1) -- (2,2,-1) -- (2,-2,-1);
\begin{scope}[canvas is yz plane at x=2,rotate=-90]
\node[circle,inner sep=0.5mm,fill,label=above:{$A$}] at (0,0.5){};
\node[circle,inner sep=0.5mm,fill,label=below:{$B$}] at (0,-0.5) {};
\end{scope}
\foreach \r in {0.25,0.5,...,2} {
	\draw (2,0.5) ++(-60:\r) arc (-60:60:\r);
	\draw (2,-0.5) ++(-60:\r) arc (-60:60:\r);
}
\draw[fill=black!10] (4,-2,-1) -- (4,-2,1) -- (4,2,1) -- (4,2,-1) -- (4,-2,-1);
\begin{scope}[xshift=4cm,yshift=2cm,rotate=-90,canvas is xy plane at z=0]
\fill[white] (0,0) rectangle (4,4);
\begin{axis}[
	width=5.575cm,
	xmin=-0.5,
	xmax=0.5,
	ticks=none
]
\addplot [samples=1000,red
]
{(cos(deg(5*pi*sin(deg(x)))))^(2)*((sin(deg(4*pi*sin(deg(x)))))/(4*pi*sin(deg(x))))^(2)};
\end{axis}
\end{scope}
\draw[thin,densely dashed,blue] (2,0) -- (6.9,0);
\end{tikzpicture}

    \end{figure}
\end{frame}
\begin{frame}
    \frametitle{Классическая логика}
    \begin{itemize}[<+->]
    \item Классическое описание:
        \begin{itemize}[<+->]
        \item Событие $A$ --- фотон прошёл через щель $A$
        \item Событие $B$ --- фотон прошёл через щель $B$
        \item Событие $C (x)$ --- фотон попал на фотопластину в точке $x$
        \item Свойство операций <<и>>, <<или>> в классической логике:
        $$ w \left \{C (x)\,\text{и}\,\left[A\;\text{или}\;B\right] \right \}
        = 
        w \left \{ \left [C (x)\,\text {и}\,A \right ]\,\text {или}\,
            \left [C (x)\,\text {и}\,B \right ] \right \} $$
        \end{itemize}
    \item Квантовое описание:
        \begin{itemize}[<+->]
        \item
        $ w \left \{C (x)\,\text{и}\,\left[A\;\text{или}\;B\right] \right \}$
        --- интерференция
        \item
        $ w \left \{ \left [C (x)\,\text {и}\,A \right ]\,\text {или}\,
            \left [C (x)\,\text {и}\,B \right ] \right \} $
        --- дифракция
        \item[ ]
        $$ w \left \{C (x)\,\text{и}\,\left[A\;\text{или}\;B\right] \right \}
        \ne 
        w \left \{ \left [C (x)\,\text {и}\,A \right ]\,\text {или}\,
            \left [C (x)\,\text {и}\,B \right ] \right \} $$
        \end{itemize}
    \end{itemize}
\end{frame}


\section{Квантовая теория поля}
\begin{frame}{Содержание раздела}
    \tableofcontents[currentsection, subsectionstyle=show/show/hide]
\end{frame}
\subsection{Дискретные симметрии}
\begin{frame}
    \frametitle{$\mathcal{P}$-, $\mathcal{T}$-, и $\mathcal{C}$- симметрии}
    \begin{itemize}[<+->]
        \item Опеределение преобразований:
        \begin{itemize}[<+->]
        \item[] $\hat{\mathcal{P}} \ket{\psi(\vec{x})} = \ket{\psi(-\vec{x})}$
        \item[] $\hat{\mathcal{T}} \ket{\psi(t)} = \ket{\psi(-t)}$
        \item[] $q_i$ --- набор зарядов,
        $\hat{\mathcal{C}} \ket{\psi(q_i)} = \ket{\psi(-q_i)}$
        \end{itemize}
    \item Собственные состояния:
        \begin{itemize}[<+->]
        \item[]
        $\hat{\mathcal{P}} \ket{\psi(\vec{x})} = \pm\ket{\psi(\vec{x})}$
        \item[]
        $\hat{\mathcal{T}} \ket{\psi(t)} = \pm\ket{\psi(t)}$
        \item[]
        $\hat{\mathcal{C}} \ket{\psi(q_i=0)} = \pm\ket{\psi}$
        \end{itemize}
    \item $\mathcal{CPT}$-теорема:
        \begin{itemize}[<+->]
        \item
        ${\mathcal{C}}{\mathcal{P}}{\mathcal{T}} = 1$
        \end{itemize}
    \end{itemize}
\end{frame}

\subsection{$\mathcal{CP}$ симметрия}
\begin{frame}
    \frametitle{Пи- и K- мезоны}
    \begin{itemize}[<+->]
    \item Пионы ($\pi$-мезоны)
        \begin{itemize}[<+->]
        \item $\pi^+$, $\pi^0$, и $\pi^-$ --- частицы-переносчики ядерного
        взаимодействия
        \item Масса $m (\pi) \approx 270\; m (\text{e})$
        \item Время жизни $\tau (\pi^\pm) \approx 2 \cdot 10^{-8}$~с,
            $\tau (\pi^0) \approx 10^{-16}$~с
        \item $\hat{\mathcal{P}} \ket{\pi^0} = - \ket{\pi^0}$
        \item $\hat{\mathcal{P}} \ket{\pi^+\pi^-,\;l} = (-1)^l
            \ket{\pi^+ \pi^-,\;l}$
        \end{itemize}
    \item Каоны (K-мезоны)
        \begin{itemize}[<+->]
        \item Масса $m (\text{K}) \approx 970\; m (\text{e})$
        \item Время жизни $\tau (\text{K}^\pm) \approx 10^{-8}$~с
        \item Нейтральные каоны несут скрытый заряд, поэтому
            $\overline{\text{K}^0} \ne \text {K}^0$
        \item Собственные комбинации для операторов чётности:
            $\text{K}^0 \pm \overline{\text{K}^0} =
            \text{K}_\text{S},\;\text{K}_\text{L}$
        \item
            $\text{K}_\text{S}\to\pi^+\pi^-$ ($\tau \approx 10^{-10}$~с,
            $\mathcal{CP} = 1$)
        \item
            $\text{K}_\text{L}\to\pi^+\pi^-\pi^0$ ($\tau \approx 5\cdot10^{-8}$~с,
            $\mathcal{CP} = -1$)
        \item
            Осцилляции нейтральных каонов: $\overline{\text{K}^0} \to \text{K}^0$
        \item
            Регенерация $\text{K}_\text{S}$ в веществе
        \item
            Нарушение $\mathcal{CP}$-симметрии (а, следовательно, и нарушение
            $\mathcal{T}$-симметрии)
        \end{itemize}
    \end{itemize}
\end{frame}

\subsection{Стандартная модель}
\begin{frame}
    \frametitle{Элементарные частицы}
    \begin{figure}
    \includegraphics[
        height=0.8\textheight,
        width=\textwidth,
        keepaspectratio,
        ]{sm}
    \end{figure}
\end{frame}

\subsection{Экспериментальные установки}
\begin{frame}
    \frametitle{$\text{e}^+\text{e}^-$ коллайдер KEKB}
    \begin{figure}
    \includegraphics[
        height=0.8\textheight,
        width=\textwidth,
        keepaspectratio,
        ]{KEKB}
    \end{figure}
\end{frame}

\begin{frame}
    \frametitle{Детектор Белль}
    \begin{columns}
        \begin{column}{0.5\textwidth}
            \begin{figure}
            \includegraphics[
            height=0.9\textheight,
            width=\textwidth,
            keepaspectratio,
            ]{belle-side-view}
            \end{figure}
        \end{column}
        \begin{column}{0.5\textwidth}
            \begin{figure}
            \includegraphics[
            height=0.9\textheight,
            width=\textwidth,
            keepaspectratio,
            ]{belle-collab}
            \end{figure}
        \end{column}
    \end{columns}
\end{frame}
%%%
\begin{frame}
    \frametitle{Дрейфовая камера}
    \begin{columns}
        \begin{column}{0.5\textwidth}
            \begin{figure}
            \includegraphics[
            height=0.7\textheight,
            width=0.9\textwidth,
            keepaspectratio,
            ]{belle-cdc}
            \end{figure}
        \end{column}
        \begin{column}{0.5\textwidth}
            \begin{figure}
            \includegraphics[
            height=0.9\textheight,
            width=0.85\textwidth,
            keepaspectratio,
            ]{belle-ed}
            \caption{Event display}
            \end{figure}
        \end{column}
    \end{columns}
\end{frame}
%%%
\begin{frame}
    \frametitle{Событие в детекторе}
    \begin{columns}
        \begin{column}{0.5\textwidth}
            \begin{figure}
            \includegraphics[
            height=0.95\textheight,
            width=0.9\textwidth,
            keepaspectratio,
            ]{belle-ed-xy}
            \caption{$\mathcal{XY}$-проекция}
            \end{figure}
        \end{column}
        \begin{column}{0.5\textwidth}
            \begin{figure}
            \includegraphics[
            height=0.95\textheight,
            width=0.9\textwidth,
            keepaspectratio,
            ]{belle-ed-yz}
            \caption{$\mathcal{YZ}$-проекция}
            \end{figure}
        \end{column}
    \end{columns}
\end{frame}
%%%
\begin{frame}
    \frametitle{Нарушение $\mathcal{T}$-симметрии}
    \begin{figure}
    \includegraphics[
        height=0.8\textheight,
        width=\textwidth,
        keepaspectratio,
        ]{Tviolation}
    \end{figure}
\end{frame}


\end{document}
