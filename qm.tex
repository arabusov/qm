\documentclass[t,aspectratio=169]{beamer}

\usetheme{Goettingen}
\usepackage[T2A]{fontenc}
\usepackage[utf8]{inputenc}
\usepackage[russian]{babel}

\usepackage{amsmath}
\usepackage{bbm}
\usepackage{verbatim}
\usepackage{multimedia}
\usepackage[absolute,overlay]{textpos}

\usepackage{braket}
\usepackage{amssymb}
\usepackage{bbold}

\usepackage{pgfplots}
\usetikzlibrary{3d}
\usepackage{tikz}
\usepackage{tikz-feynman}
\usepackage{tikz-3dplot}
\usetikzlibrary{arrows,automata,backgrounds,calendar,chains,matrix,mindmap,
    patterns,petri,shadows,shapes.geometric,shapes.misc,spy,trees,calc}

\newcommand{\Sp}[1]{\mathbbm {Sp} \left\{ {#1} \right\} }
\newcommand\norm[1]{\left\lVert#1\right\rVert}
\setbeamertemplate{navigation symbols}{}
\setbeamertemplate{caption}{\raggedright\insertcaption\par}
\usepackage{url}
\usefonttheme[onlymath]{serif}
\graphicspath{{./images/}}

\author[А. Рабусов]{Андрей Рабусов}
\title[Квантовая физика]{Квантовая физика}
\subtitle{Введение}
\date{27 декабря 2020}
\begin{document}
\begin{frame}
    \maketitle
\end{frame}
\section{Квантовая механика}
\subsection{Введение}
\begin{frame}
    \frametitle{Математическая база}
    \begin{itemize}[<+->]
    \item
        Поле комплексных чисел $\mathbb{C}$
        \begin{itemize}[<+->]
        \item $i^2 = -1$
        \item $z = x + i\,y$, $z^\ast = x - i\, y$
        \item $e^{i \varphi} = \cos \varphi + i \sin \varphi$
        \end{itemize}
    \item
        Гильбертово пространство $\mathcal{H}$
        над полем комплексных чисел $\mathbb{C}$
        \begin{itemize}[<+->]
        \item Линейное векторное пространство
        \item
            Определено скалярное произведение для любой пары
            векторов $\bf{u}$ и $\bf{v}$:
            $$(\bf{u}^\dagger, \bf{v}) \in \mathbb{C}$$
        \item
            Скалярное произведение порождает норму, например
            $$\norm {\bf{u}}^2 = (\bf {u}^\dagger, \bf{u})$$
        \end{itemize}
    \end{itemize}
\end{frame}

\subsection{Статические постулаты}
\begin{frame}
    \frametitle{Постулаты $1$--$2$}
    \begin{itemize}[<+->]
    \item[П1] Чистое состояние системы $\ket{\psi}$ описывается
    вектором гильбертова пространства $\mathcal{H}$ над $\mathbb{C}$
    \item[П2] Чистое состояние $\ket{\psi}$ задаётся
    только направлением вектора из $\mathcal{H}$
        \begin{itemize}[<+->]
        \item Можно выбрать любую ненулевую нормировку, например
        $$ \braket{\psi|\psi} = 1 $$
        \item Эволюция замкнутой системы унитарна
        $$ \ket{\psi (t)} = U (t, t_0) \ket{\psi (t_0)}$$
        $$ U^\dagger (t, t_0) \; U (t, t_0) = \hat{\mathbb{1}} $$
        \end{itemize}
    \end{itemize}
\end{frame}
\begin{frame}
    \frametitle{Принцип суперпозиции (постулат 3)}
    \begin{itemize}[<+->]
    \item Положим, что квантовая система до измерения в состоянии
    $\ket{\psi}$
    \item После многократных измерений при помощи макроприбора
    $\ket{\psi} \to \ket{\varphi_i}$
    \item $\ket{\varphi_i}$ различимы при помощи макроприбора, т.~е.
    $$ \braket {\varphi_i| \varphi_j} = \delta_{ij} = 
        \begin{cases}
        0 & \text {if $i \ne j $} \\
        1 & \text {if $i = j $}
        \end{cases}
    $$
    \item
    $ \ket {\psi} = \sum\limits_i c_i \ket {\varphi_i},\;c_i \in \mathbb{C} $,
    $ c_i = \braket{\varphi_i | \psi} $
    \item[$\hat{\mathbb{P}}_{\varphi_i}$] Проекционный оператор
    $\hat{\mathbb{P}}_{\varphi_i} = \ket{\phi_i}\bra{\phi_i}$
    $$\hat{\mathbb{P}}_{\varphi_i}\ket{\psi} = \ket{\phi_i}$$
    \item $\sum\limits_i \hat{\mathbb{P}}_{\varphi_i} = \hat{\mathbb{1}}$
    \end{itemize}
\end{frame}
\begin{frame}
    \frametitle{Постулаты $4$--$5$}
    \begin{itemize}[<+->]
    \item[П4] Физический смысл $c_i$ (Макс Борн)
        \begin{itemize}[<+->]
        \item Пусть во время измерения $\ket{\psi}\to\ket{\varphi_i}$
        \item Условная вероятность $w_i$ обнаружить систему в состоянии $i$:
        $$ w_i = \abs{ c_i }^2 = \braket {\psi|\varphi_i}
        \braket {\varphi_i|\psi} =
        \Sp {\hat{\mathbb{P}}_\psi \hat{\mathbb{P}}_{\varphi_i}}$$
        \end{itemize}
    \item[П5] Среднее наблюдаемого
        \begin{itemize}[<+->]
        \item Набор наблюдаемых $\left \{ a_i \right \}$
        (спектр, $a_i \in \mathbb{R}$)
        \item
        $\hat{\mathbb{A}} \ket{ a_i } = a_i \ket{ a_i }$,
        $\hat{\mathbb{A}}$ --- эрмитовский оператор, $\ket {a_i}$ ---
        собственные состояния оператора $\hat{\mathbb{A}}$
        \item Среднее значение наблюдаемой $a$ по состоянию $\ket{\psi}$:
        $$ \langle a \rangle = \braket {\psi | \hat{\mathbb{A}} | \psi} = 
        \Sp {\hat{\mathbb{A}} \hat{\mathbb{P}}_\psi} $$

        \end{itemize}
    \end{itemize}
\end{frame}

\subsection{Двухщелевой опыт}
\begin{frame}
    \frametitle{Волновое поведение}
    \begin{figure}
    % Taken from here:
% https://tex.stackexchange.com/questions/469109/\
% how-to-change-arrowheads-to-lie-on-a-plane
\begin{tikzpicture}[scale=1.25,every node/.append style={transform shape}]
\foreach \x in {-1,-0.75,...,0} {
	\draw (\x,-1) -- (\x,1);
}
\draw[fill=black!10] (0.5,-2,-1) -- (0.5,-2,1) -- (0.5,2,1) -- (0.5,2,-1) -- (0.5,-2,-1);
\fill (0.5,0,0) circle (0.05);
\foreach \r in {0.25,0.5,...,1.75} {
	\draw (0.5,0) ++(-60:\r) arc (-60:60:\r);
}
\draw[fill=black!10] (2,-2,-1) -- (2,-2,1) -- (2,2,1) -- (2,2,-1) -- (2,-2,-1);
%\fill (2,0.5) circle (0.05) (2,-0.5) circle (0.05);
\foreach \r in {0.25,0.5,...,2} {
	\draw (2,0.5) ++(-60:\r) arc (-60:60:\r);
	\draw (2,-0.5) ++(-60:\r) arc (-60:60:\r);
}
\draw[fill=black!10] (4,-2,-1) -- (4,-2,1) -- (4,2,1) -- (4,2,-1) -- (4,-2,-1);
%       LABELLING
\begin{scope}[canvas is yz plane at x=2,rotate=-90]
\node[circle,inner sep=0.5mm,fill,label=above:{S${}_1$}] at (0,0.5){};
\node[circle,inner sep=0.5mm,fill,label=below:{S${}_2$}] at (0,-0.5) {};
\draw[dash pattern=on 1.5pt off 1pt,thin] (-0.5,0.5) -- (0,0.5)
	(-0.5,-0.5) -- (0,-0.5);
	\pgflowlevelsynccm          
	\draw[|<->|] (-0.5,0.5) -- (-0.5,-0.5) node[midway,right=-0.1cm] {d};
\end{scope}
\begin{scope}[canvas is yz plane at x=0.5,rotate=-90]
\node[below left=-0.1cm] at (0,0) {S${}_0$};
\end{scope}
\begin{scope}[xshift=4cm,yshift=2cm,rotate=-90,canvas is xy plane at z=0]
\fill[white] (0,0) rectangle (4,4);
\begin{axis}[
	width=5.575cm,
	xmin=-0.5,
	xmax=0.5,
	ticks=none
]
\addplot [samples=1000,blue
]
{(cos(deg(5*pi*sin(deg(x)))))^(2)*((sin(deg(4*pi*sin(deg(x)))))/(4*pi*sin(deg(x))))^(2)};
\end{axis}
\end{scope}
\draw[thin,densely dashed,blue] (2,0) -- (6.9,0);
\draw[thin,densely dashed,blue] (2,0) -- +(15:2.5);
\draw[thin,densely dashed,blue] (2,0) -- +(-15:2.5);
\draw[thin,densely dashed,blue] (2,0) -- +(32:2.5);
\draw[thin,densely dashed,blue] (2,0) -- +(-32:2.5);
\end{tikzpicture}

    \end{figure}
\end{frame}
\begin{frame}
    \frametitle{Дополнительный детектор у щели $A$}
    \begin{figure}
    % Taken from here:
% https://tex.stackexchange.com/questions/469109/\
% how-to-change-arrowheads-to-lie-on-a-plane
\begin{tikzpicture}[scale=1.25]%,every node/.append style={transform shape}]
\draw[color=white] (-2,-2) -- (8, 3);
\only<1->{
\fill (0.5,0,0) circle (0.05);
\foreach \r in {0.25,0.5,...,1.75} {
	\draw (0.5,0) ++(-60:\r) arc (-60:60:\r);
\node[below left=-0.1cm] at (0,0) {$S$};
}
}
\only<2->{
\draw[fill=black!10] (2,-2,-1) -- (2,-2,1) -- (2,2,1) -- (2,2,-1) -- (2,-2,-1);
\begin{scope}[canvas is yz plane at x=2,rotate=-90]
\node[circle,inner sep=0.5mm,fill,label=above:{$A\;D$}] at (0,0.5){};
\node[circle,inner sep=0.5mm,fill,label=below:{$B$}] at (0,-0.5) {};
\end{scope}
}
\only<3->{
\foreach \r in {0.25,0.5,...,2} {
	\draw (2,0.5) ++(-60:\r) arc (-60:60:\r);
	\draw (2,-0.5) ++(-60:\r) arc (-60:60:\r);
}
}
\only<4->{
\draw[fill=black!10] (4,-2,-1) -- (4,-2,1) -- (4,2,1) -- (4,2,-1) -- (4,-2,-1);
}
\only<5->{
\begin{scope}[xshift=4cm,yshift=2cm,rotate=-90,canvas is xy plane at z=0]
\fill[white] (0,0) rectangle (4,4);
\begin{axis}[
	width=5.575cm,
	xmin=-0.5,
	xmax=0.5,
	ticks=none
]
\addplot [samples=1000,blue
]
%{(cos(deg(5*pi*sin(deg(x)))))^(2)*((sin(deg(4*pi*sin(deg(x)))))/(4*pi*sin(deg(x))))^(2)};
{((1/((x-0.2)^(2)+0.001))+(1/((x+0.2)^(2)+0.001)))};
\end{axis}
\end{scope}
\draw[thin,densely dashed,blue] (2,0) -- (6.9,0);
}
\end{tikzpicture}

    \end{figure}
\end{frame}
\begin{frame}
    \frametitle{Волновое поведение}
    \begin{figure}
    % Taken from here:
% https://tex.stackexchange.com/questions/469109/\
% how-to-change-arrowheads-to-lie-on-a-plane
\begin{tikzpicture}[scale=1.25]%,every node/.append style={transform shape}]
\draw[color=white] (-2,-2) -- (8, 3);
\fill (0.5,0,0) circle (0.05);
\foreach \r in {0.25,0.5,...,1.75} {
	\draw (0.5,0) ++(-60:\r) arc (-60:60:\r);
\node[below left=-0.1cm] at (0,0) {$S$};
}
\draw[fill=black!10] (2,-2,-1) -- (2,-2,1) -- (2,2,1) -- (2,2,-1) -- (2,-2,-1);
\begin{scope}[canvas is yz plane at x=2,rotate=-90]
\node[circle,inner sep=0.5mm,fill,label=above:{$A$}] at (0,0.5){};
\node[circle,inner sep=0.5mm,fill,label=below:{$B$}] at (0,-0.5) {};
\end{scope}
\foreach \r in {0.25,0.5,...,2} {
	\draw (2,0.5) ++(-60:\r) arc (-60:60:\r);
	\draw (2,-0.5) ++(-60:\r) arc (-60:60:\r);
}
\draw[fill=black!10] (4,-2,-1) -- (4,-2,1) -- (4,2,1) -- (4,2,-1) -- (4,-2,-1);
\begin{scope}[xshift=4cm,yshift=2cm,rotate=-90,canvas is xy plane at z=0]
\fill[white] (0,0) rectangle (4,4);
\begin{axis}[
	width=5.575cm,
	xmin=-0.5,
	xmax=0.5,
	ticks=none
]
\addplot [samples=1000,red
]
{(cos(deg(5*pi*sin(deg(x)))))^(2)*((sin(deg(4*pi*sin(deg(x)))))/(4*pi*sin(deg(x))))^(2)};
\end{axis}
\end{scope}
\draw[thin,densely dashed,blue] (2,0) -- (6.9,0);
\end{tikzpicture}

    \end{figure}
\end{frame}


\end{document}
