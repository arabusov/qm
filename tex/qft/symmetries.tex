\subsection{Дискретные симметрии}
\begin{frame}
    \frametitle{$\mathcal{P}$-, $\mathcal{T}$-, и $\mathcal{C}$- симметрии}
    \begin{itemize}[<+->]
        \item Опеределение преобразований:
        \begin{itemize}[<+->]
        \item[] $\hat{\mathcal{P}} \ket{\psi(\vec{x})} = \ket{\psi(-\vec{x})}$
        \item[] $\hat{\mathcal{T}} \ket{\psi(t)} = \ket{\psi(-t)}$
        \item[] $q_i$ --- набор зарядов,
        $\hat{\mathcal{C}} \ket{\psi(q_i)} = \ket{\psi(-q_i)}$
        \end{itemize}
    \item Собственные состояния:
        \begin{itemize}[<+->]
        \item[]
        $\hat{\mathcal{P}} \ket{\psi(\vec{x})} = \pm\ket{\psi(\vec{x})}$
        \item[]
        $\hat{\mathcal{T}} \ket{\psi(t)} = \pm\ket{\psi(t)}$
        \item[]
        $\hat{\mathcal{C}} \ket{\psi(q_i=0)} = \pm\ket{\psi}$
        \end{itemize}
    \item $\mathcal{CPT}$-теорема:
        \begin{itemize}[<+->]
        \item
        ${\mathcal{C}}{\mathcal{P}}{\mathcal{T}} = 1$
        \end{itemize}
    \end{itemize}
\end{frame}
