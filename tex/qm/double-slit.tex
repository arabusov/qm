\subsection{Двухщелевой опыт}
\begin{frame}
    \frametitle{Интерференция}
    \begin{figure}
    % Taken from here:
% https://tex.stackexchange.com/questions/469109/\
% how-to-change-arrowheads-to-lie-on-a-plane
\begin{tikzpicture}[scale=1.25,every node/.append style={transform shape}]
\foreach \x in {-1,-0.75,...,0} {
	\draw (\x,-1) -- (\x,1);
}
\draw[fill=black!10] (0.5,-2,-1) -- (0.5,-2,1) -- (0.5,2,1) -- (0.5,2,-1) -- (0.5,-2,-1);
\fill (0.5,0,0) circle (0.05);
\foreach \r in {0.25,0.5,...,1.75} {
	\draw (0.5,0) ++(-60:\r) arc (-60:60:\r);
}
\draw[fill=black!10] (2,-2,-1) -- (2,-2,1) -- (2,2,1) -- (2,2,-1) -- (2,-2,-1);
%\fill (2,0.5) circle (0.05) (2,-0.5) circle (0.05);
\foreach \r in {0.25,0.5,...,2} {
	\draw (2,0.5) ++(-60:\r) arc (-60:60:\r);
	\draw (2,-0.5) ++(-60:\r) arc (-60:60:\r);
}
\draw[fill=black!10] (4,-2,-1) -- (4,-2,1) -- (4,2,1) -- (4,2,-1) -- (4,-2,-1);
%       LABELLING
\begin{scope}[canvas is yz plane at x=2,rotate=-90]
\node[circle,inner sep=0.5mm,fill,label=above:{S${}_1$}] at (0,0.5){};
\node[circle,inner sep=0.5mm,fill,label=below:{S${}_2$}] at (0,-0.5) {};
\draw[dash pattern=on 1.5pt off 1pt,thin] (-0.5,0.5) -- (0,0.5)
	(-0.5,-0.5) -- (0,-0.5);
	\pgflowlevelsynccm          
	\draw[|<->|] (-0.5,0.5) -- (-0.5,-0.5) node[midway,right=-0.1cm] {d};
\end{scope}
\begin{scope}[canvas is yz plane at x=0.5,rotate=-90]
\node[below left=-0.1cm] at (0,0) {S${}_0$};
\end{scope}
\begin{scope}[xshift=4cm,yshift=2cm,rotate=-90,canvas is xy plane at z=0]
\fill[white] (0,0) rectangle (4,4);
\begin{axis}[
	width=5.575cm,
	xmin=-0.5,
	xmax=0.5,
	ticks=none
]
\addplot [samples=1000,blue
]
{(cos(deg(5*pi*sin(deg(x)))))^(2)*((sin(deg(4*pi*sin(deg(x)))))/(4*pi*sin(deg(x))))^(2)};
\end{axis}
\end{scope}
\draw[thin,densely dashed,blue] (2,0) -- (6.9,0);
\draw[thin,densely dashed,blue] (2,0) -- +(15:2.5);
\draw[thin,densely dashed,blue] (2,0) -- +(-15:2.5);
\draw[thin,densely dashed,blue] (2,0) -- +(32:2.5);
\draw[thin,densely dashed,blue] (2,0) -- +(-32:2.5);
\end{tikzpicture}

    \end{figure}
\end{frame}
\begin{frame}
    \frametitle{Дополнительный детектор у щели $A$ --- дифракция}
    \begin{figure}
    % Taken from here:
% https://tex.stackexchange.com/questions/469109/\
% how-to-change-arrowheads-to-lie-on-a-plane
\begin{tikzpicture}[scale=1.25]%,every node/.append style={transform shape}]
\draw[color=white] (-2,-2) -- (8, 3);
\only<1->{
\fill (0.5,0,0) circle (0.05);
\foreach \r in {0.25,0.5,...,1.75} {
	\draw (0.5,0) ++(-60:\r) arc (-60:60:\r);
\node[below left=-0.1cm] at (0,0) {$S$};
}
}
\only<2->{
\draw[fill=black!10] (2,-2,-1) -- (2,-2,1) -- (2,2,1) -- (2,2,-1) -- (2,-2,-1);
\begin{scope}[canvas is yz plane at x=2,rotate=-90]
\node[circle,inner sep=0.5mm,fill,label=above:{$A\;D$}] at (0,0.5){};
\node[circle,inner sep=0.5mm,fill,label=below:{$B$}] at (0,-0.5) {};
\end{scope}
}
\only<3->{
\foreach \r in {0.25,0.5,...,2} {
	\draw (2,0.5) ++(-60:\r) arc (-60:60:\r);
	\draw (2,-0.5) ++(-60:\r) arc (-60:60:\r);
}
}
\only<4->{
\draw[fill=black!10] (4,-2,-1) -- (4,-2,1) -- (4,2,1) -- (4,2,-1) -- (4,-2,-1);
}
\only<5->{
\begin{scope}[xshift=4cm,yshift=2cm,rotate=-90,canvas is xy plane at z=0]
\fill[white] (0,0) rectangle (4,4);
\begin{axis}[
	width=5.575cm,
	xmin=-0.5,
	xmax=0.5,
	ticks=none
]
\addplot [samples=1000,blue
]
%{(cos(deg(5*pi*sin(deg(x)))))^(2)*((sin(deg(4*pi*sin(deg(x)))))/(4*pi*sin(deg(x))))^(2)};
{((1/((x-0.2)^(2)+0.001))+(1/((x+0.2)^(2)+0.001)))};
\end{axis}
\end{scope}
\draw[thin,densely dashed,blue] (2,0) -- (6.9,0);
}
\end{tikzpicture}

    \end{figure}
\end{frame}
\begin{frame}
    \frametitle{Волновое поведение}
    \begin{figure}
    % Taken from here:
% https://tex.stackexchange.com/questions/469109/\
% how-to-change-arrowheads-to-lie-on-a-plane
\begin{tikzpicture}[scale=1.25]%,every node/.append style={transform shape}]
\draw[color=white] (-2,-2) -- (8, 3);
\fill (0.5,0,0) circle (0.05);
\foreach \r in {0.25,0.5,...,1.75} {
	\draw (0.5,0) ++(-60:\r) arc (-60:60:\r);
\node[below left=-0.1cm] at (0,0) {$S$};
}
\draw[fill=black!10] (2,-2,-1) -- (2,-2,1) -- (2,2,1) -- (2,2,-1) -- (2,-2,-1);
\begin{scope}[canvas is yz plane at x=2,rotate=-90]
\node[circle,inner sep=0.5mm,fill,label=above:{$A$}] at (0,0.5){};
\node[circle,inner sep=0.5mm,fill,label=below:{$B$}] at (0,-0.5) {};
\end{scope}
\foreach \r in {0.25,0.5,...,2} {
	\draw (2,0.5) ++(-60:\r) arc (-60:60:\r);
	\draw (2,-0.5) ++(-60:\r) arc (-60:60:\r);
}
\draw[fill=black!10] (4,-2,-1) -- (4,-2,1) -- (4,2,1) -- (4,2,-1) -- (4,-2,-1);
\begin{scope}[xshift=4cm,yshift=2cm,rotate=-90,canvas is xy plane at z=0]
\fill[white] (0,0) rectangle (4,4);
\begin{axis}[
	width=5.575cm,
	xmin=-0.5,
	xmax=0.5,
	ticks=none
]
\addplot [samples=1000,red
]
{(cos(deg(5*pi*sin(deg(x)))))^(2)*((sin(deg(4*pi*sin(deg(x)))))/(4*pi*sin(deg(x))))^(2)};
\end{axis}
\end{scope}
\draw[thin,densely dashed,blue] (2,0) -- (6.9,0);
\end{tikzpicture}

    \end{figure}
\end{frame}
\begin{frame}
    \frametitle{Классическая логика}
    \begin{itemize}[<+->]
    \item Классическое описание:
        \begin{itemize}[<+->]
        \item Событие $A$ --- фотон прошёл через щель $A$
        \item Событие $B$ --- фотон прошёл через щель $B$
        \item Событие $C (x)$ --- фотон попал на фотопластину в точке $x$
        \item Свойство операций <<и>>, <<или>> в классической логике:
        $$ w \left \{C (x)\,\text{и}\,\left[A\;\text{или}\;B\right] \right \}
        = 
        w \left \{ \left [C (x)\,\text {и}\,A \right ]\,\text {или}\,
            \left [C (x)\,\text {и}\,B \right ] \right \} $$
        \end{itemize}
    \item Квантовое описание:
        \begin{itemize}[<+->]
        \item
        $ w \left \{C (x)\,\text{и}\,\left[A\;\text{или}\;B\right] \right \}$
        --- интерференция
        \item
        $ w \left \{ \left [C (x)\,\text {и}\,A \right ]\,\text {или}\,
            \left [C (x)\,\text {и}\,B \right ] \right \} $
        --- дифракция
        \item[ ]
        $$ w \left \{C (x)\,\text{и}\,\left[A\;\text{или}\;B\right] \right \}
        \ne 
        w \left \{ \left [C (x)\,\text {и}\,A \right ]\,\text {или}\,
            \left [C (x)\,\text {и}\,B \right ] \right \} $$
        \end{itemize}
    \end{itemize}
\end{frame}
