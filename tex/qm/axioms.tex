\subsection{Статические постулаты}
\begin{frame}
    \frametitle{Постулаты $1$--$2$}
    \begin{itemize}[<+->]
    \item[П1] Чистое состояние системы $\ket{\psi}$ описывается
    вектором гильбертова пространства $\mathcal{H}$ над $\mathbb{C}$
    \item[П2] Чистое состояние $\ket{\psi}$ задаётся
    только направлением вектора из $\mathcal{H}$
        \begin{itemize}[<+->]
        \item Можно выбрать любую ненулевую нормировку, например
        $$ \braket{\psi|\psi} = 1 $$
        \item Эволюция замкнутой системы унитарна
        $$ \ket{\psi (t)} = U (t, t_0) \ket{\psi (t_0)}$$
        $$ U^\dagger (t, t_0) \; U (t, t_0) = \hat{\mathbb{1}} $$
        \end{itemize}
    \end{itemize}
\end{frame}
\begin{frame}
    \frametitle{Принцип суперпозиции (постулат 3)}
    \begin{itemize}[<+->]
    \item Положим, что квантовая система до измерения в состоянии
    $\ket{\psi}$
    \item После многократных измерений при помощи макроприбора
    $\ket{\psi} \to \ket{\varphi_i}$
    \item $\ket{\varphi_i}$ различимы при помощи макроприбора, т.~е.
    $$ \braket {\varphi_i| \varphi_j} = \delta_{ij} = 
        \begin{cases}
        0 & \text {if $i \ne j $} \\
        1 & \text {if $i = j $}
        \end{cases}
    $$
    \item
    $ \ket {\psi} = \sum\limits_i c_i \ket {\varphi_i},\;c_i \in \mathbb{C} $
    \item[$\hat{\mathbb{P}}_{\varphi_i}$] Проекционный оператор
    $\hat{\mathbb{P}}_{\varphi_i} = \ket{\phi_i}\bra{\phi_i}$
    $$\hat{\mathbb{P}}_{\varphi_i}\ket{\psi} = \ket{\phi_i}$$
    \item $\sum\limits_i \hat{\mathbb{P}}_{\varphi_i} = \hat{\mathbb{1}}$
    \end{itemize}
\end{frame}
